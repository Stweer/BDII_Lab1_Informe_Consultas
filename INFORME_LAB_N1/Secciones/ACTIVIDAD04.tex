\section{Actividad No 04 – Restricci\'on y Ordenamiento} 
		
\begin{enumerate}[1.]
	\item Debido a problemas con el presupuesto, el departamento de Recursos Humanos requiere un reporte que muestre los apellidos (last\_name) y salarios (salary) de todos los empleados que ganen más de \$ 12,000.
	\\ 

	\item Asimismo se requiere realizar una consulta que muestre los apellidos (last\_name) y el n\'umero de departamento (department\_id) para los empleados que tengan numero (employee\_id) 176.
	\\ 

	\item El departamento de Recursos Humanos necesita determinar los mayores y menores sueldos, modificar la consulta del  ítem 4.1. para mostrar el apellido y salario de cada empleado cuyo sueldo no est\'e en el rango de \$ 5,000 a \$ 12,000.
	\\ 

	\item Crear un reporte que muestre los apellidos (last\_name), puesto (job\_id) y fecha de contrataci\'on (hire\_date), de los empleados que apellidan ‘Matos’ y ‘Taylor’, asimismo presentar el reporte ordenado ascendentemente por fecha de contrataci\'on.
	\\ 

	\item Mostrar los apellidos (last\_name) y n\'umero de departamento (departamento\_id) de todos los empleados que pertenezcan a los departamentos 20 o 50 en orden alfab\'etico ascendente por el apellido.
	\\ 
	
	\item Modificar el reporte del ítem 4.1. para mostrar los apellidos y salarios de los empleados que tengan un salario entre los \$ 5,000 a \$ 12,000 y pertenezcan a los números de departamento 20 o 50. Asimismo etiquetar las cabeceras de los resultados con los alias Empleado y Salario Mensual respectivamente.
	\\ 

	\item El departamento de Recursos Humanos necesita un listado de apellidos (last\_name) y fecha de contrataci\'on (hire\_date) de todos los empleados que fueron contratados el año 1994.
	\\ 

	\item Crear un reporte que muestre los apellidos (last\_name) y puesto (job\_id) de todos los empleados que no tengan un administrador (manager).
	\\ 

	\item Crear un reporte para mostrar los apellidos (last\_name), salario (salary) y \% de comisión (commission\_pct). Ordenar los datos por salario y comisión de manera descendente, utilizar la opción numérica de la cláusula ORDER BY.
	\\ 

	\item El personal del departamento de Recursos Humanos desea tener mayor flexibilidad con los reportes hechos. Por ejemplo se requiere un reporte de los apellidos (last\_name) y salarios (salary) de todos los empleados que tengan un salario mayor a un monto que el personal de Recursos Humanos ingresará. Probar con el valor \$ 12,000.
	\\ 

	\item El departamento de Recursos Humanos requiere extraer reporte basados en el Administrador (manager\_id). Se requiere crear una consulta que pregunte al usuario por el Administrador (manager\_id) y genere un reporte con los números de empleado (employee\_id), apellidos (last\_name), salarios (salary) y numero de departamento de los empleados que este Administrador tiene a su cargo. Adicionalmente también se desea tener la habilidad de ordenar este reporte en base a una determinada columna. Probar con los siguientes valores:
	\\Administrador (manager\_id) = 103, ordenado por Apellido (last\_name)
	\\Administrador (manager\_id) = 201, ordenado por Salario (salary)
	\\Administrador (manager\_id) = 124, ordenado por No de Empleado (employee\_id)
	\\ 

	\item Generar un listado de apellidos (last\_name) de todos los empleados que tengan la letra ‘a’ en la tercera letra de su apellido.
	\\

	\item Mostrar los apellidos (last\_name) de todos los empleados que tengan tanto la letra ‘a’ como la letra ‘e’ en su apellido.
	\\ 

	\item Mostrar los apellidos (last\_name), puestos (job\_id) y salario (salary) de todos los empleados que sean Representantes de Ventas (SA\_REP) o Responsables de Inventario (ST\_CLERK) y cuyos salarios no sean iguales a \$ 2,500, \$ 3,500 o \$ 7,000.
	\\ 

	\item Modificar el reporte del ítem 4.6 y mostrar adicionalmente los datos de comisión (commission\_pct) de todos los empleados que solamente el 20\% de comisi\'on.
	\\ 

\end{enumerate}
